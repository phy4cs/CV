%%%%%%%%%%%%%%%%%%%%%%%%%%%%%%%%%%%%%%%%%
% Classicthesis-Styled CV
% LaTeX Template (modified)
% Version 1.0 (22/2/13)
%
% This template has been downloaded from:
% http://www.LaTeXTemplates.com
%
% Original author:
% Alessandro Plasmati
%
% License:
% CC BY-NC-SA 3.0 (http://creativecommons.org/licenses/by-nc-sa/3.0/)
%
% THIS IS THE ACADEMIC VERSION
%
%%%%%%%%%%%%%%%%%%%%%%%%%%%%%%%%%%%%%%%%%

%----------------------------------------------------------------------------------------
%	PACKAGES AND OTHER DOCUMENT CONFIGURATIONS
%----------------------------------------------------------------------------------------

\documentclass{scrartcl}

\reversemarginpar % Move the margin to the left of the page 

\newcommand{\MarginText}[1]{\marginpar{\raggedleft\bfseries\large#1}} % New command defining the margin text style

\usepackage[nochapters]{classicthesis} % Use the classicthesis style for the style of the document
\usepackage[LabelsAligned]{currvita} % Use the currvita style for the layout of the document

\renewcommand{\cvheadingfont}{\LARGE\color{Maroon}} % Font color of your name at the top

\usepackage{hyperref} % Required for adding links	and customizing them
\hypersetup{colorlinks, breaklinks, urlcolor=Maroon, linkcolor=Maroon} % Set link colors
\usepackage{multicol}

\newlength{\datebox}\settowidth{\datebox}{Summer 2015} % Set the width of the date box in each block

\newcommand{\NewEntry}[3]{\noindent\hangindent=1em\hangafter=0 \parbox{\datebox}{\small \textit{#1}}\hspace{0em} #2 #3 % Define a command for each new block - change spacing and font sizes here: #1 is the left margin, #2 is the italic date field and #3 is the position/employer/location field
\vspace{0.5em}} % Add some white space after each new entry

\newcommand{\Description}[1]{\hangindent=1em\hangafter=0\noindent\footnotesize{#1}\par\normalsize\vspace{1em}} % Define a command for descriptions of each entry - change spacing and font sizes here

\addtolength{\evensidemargin}{-1in}
\addtolength{\textwidth}{1in}

%----------------------------------------------------------------------------------------

\begin{document}

\thispagestyle{empty} % Stop the page count at the bottom of the first page

%----------------------------------------------------------------------------------------
%	NAME AND CONTACT INFORMATION SECTION
%----------------------------------------------------------------------------------------

\begin{cv}{\spacedallcaps{Christopher Symonds}}\vspace{1.5em} % Your name

\noindent\spacedallcaps{Personal Information}\vspace{0.5em} % Personal information heading

\NewEntry{address}{1 Hawksworth Grove, Leeds, LS5 3NB} % Email address

\NewEntry{email}{\href{mailto:chris.c.symonds@gmail.com}{chris.c.symonds@gmail.com}} % Email address

\NewEntry{phone}{(H) (0113) 228 0340\ \ $\cdotp$\ \ (M) (0785) 326 1340} % Phone number(s)

\vspace{1em} % Extra white space between the personal information section and goal

\noindent\spacedallcaps{Profile}\vspace{1em} % Goal heading, could be used for a quotation or short profile instead

\Description{
I am a doctoral student in the final stages of my PhD, expecting to submit my thesis in late august/early september. I am a proactive researcher with experience in quantum molecular dynamics methods, the study of ultrafast chemical processes and theoretical method development.}\vspace{0.5em} % Goal text

%----------------------------------------------------------------------------------------
%	EDUCATION
%----------------------------------------------------------------------------------------

\noindent\spacedallcaps{Education}\vspace{1em}

\NewEntry{2012-Present}{\MarginText{PhD in Computational Chemistry}\small \textbf{University of Leeds}}{\hfill \textit{School of Chemistry}}

\Description{Thesis: \textit{Simulation of Quantum Effects by use of the Coupled Coherent States Family of Methods}\newline
	
	{Description: My work during this project has focussed on applying the Coupled Coherent States (CCS) approach and Multi-Configurational Ehrenfest (MCE) method to various systems, most notably using CCS to simulate high harmonic generation in a single electron system, and discovering and compensating for differences in numerical behaviour of simulations using the two formulations of the MCE method. So far this has resulted in one publication (Phys. Rev. A, 91, 023427 (2015)), with another two in preparation. This work involved creating and expanding a large, modular and adaptable program using Fortran95 which was capable of using different equation sets and Hamiltonians without significant rewrites.}\newline
	Advisor: Prof.~Dmitry \textsc{Shalashilin}, \href{mailto:D.Shalashilin@leeds.ac.uk}{D.Shalashilin@leeds.ac.uk}}

%------------------------------------------------

\NewEntry{2008-2012}{\MarginText{Integrated MPhys in Physics}\small\textbf{University of Leeds}}{\hfill \textit{School of Physics and Astronomy}}

\Description{Classification - \textit{1st Class (Hons)}\newline
	
	\noindent{Description: My final year MPhys project involved investigating the properties and uses of microwave cavities, both experimentally and through simulation, with an emphasis on cavity-QED for quantum computation purposes}.\newline
	Advisor: Prof.~Ben \textsc{Varcoe}, \href{mailto:B.Varcoe@leeds.ac.uk}{B.Varcoe@leeds.ac.uk}}

%------------------------------------------------

\NewEntry{2007-2008}{\MarginText{Foundation Year}\small\textbf{University of Manchester}}

\Description{Passed with 85\% grade, equivalent to AAA at A level in Maths, Further Maths and Physics}
%------------------------------------------------

\NewEntry{2001-2004}{\MarginText{A-Levels}\small\textbf{Notre Dame RC Sixth Form College}}

\Description{Computing -- B, Maths -- C, Physics -- C}

\vspace{1em} % Extra space between major sections


%----------------------------------------------------------------------------------------
%	Research Interests
%----------------------------------------------------------------------------------------

\noindent\spacedallcaps{Research Interests}\vspace{1em} % Goal heading, could be used for a quotation or short profile instead

\Description{My main research interests remain in the investigation of fast processes using quantum mechanics, use of quantum optics and in the development of theoretical methods.}

%----------------------------------------------------------------------------------------
%	Teaching Experience
%----------------------------------------------------------------------------------------

\noindent\spacedallcaps{Teaching Experience}\vspace{1em} % Goal heading, could be used for a quotation or short profile instead

\Description{\textbullet \ \ I help with teaching in the undergraduate labs, which involves interview-based marking and helping students with software based or mathematical queries.\newline
\textbullet \ \ Help also with a weekly workshop for the "Physics for Chemists" first year module which involves helping students with simple physics and mathematics problems.\newline
\textbullet \ \ In the final year of my undergraduate degree I took the module ``Physics in Schools'', which placed me in a Leeds high school one afternoon a week, where it was my job to run an after school science club for age 11-13 students, performing demonstration experiments, teaching the science behind the experiments and planning and supervising more hands on sessions.}

%----------------------------------------------------------------------------------------
%	Skills
%----------------------------------------------------------------------------------------

\noindent\spacedallcaps{Skills}\vspace{1em}

\Description{\MarginText{Computational and Programming Skills}\textbullet \ \ Over the course of my PhD I have learnt to be proficient in Fortran, bash and the use of OpenMP, git, gnuplot, OriginPro and the SunGrid Engine\newline
\textbullet \ \ I have attended workshops on program optimisation, profiling, and systematic debugging as well as a week long summer school on high performance computing run by the Hartree centre which covered topics from system architecture to parallel programming.}

%------------------------------------------------

\Description{\MarginText{Communication Skills}\textbullet \ \ Over the course of my PhD I have presented my work at regional, national and international conferences, as well as at internal seminars and conferences within the department.\newline
\textbullet \ \ Through this experience I have learnt how to explain complex ideas in terms accessible to non-specialists and how to explain and defend my work succinctly and confidently.\newline
\textbullet \ \ My written communication skills are also well developed, and in addition to my publications and preparing various reports during my time at Leeds I regularly help with editing the work of others in the group.}

%-------------------------------------------------

\Description{\MarginText{Self Management}
\textbullet \ \ Over the course of work towards my PhD, it has been necessary to set my own targets and deadlines, set weekly and monthly goals and manage my time properly to meet my targets.\newline
\textbullet \ \ Have been able to balance the heavy workload of both my undergraduate and postgraduate study with my responsibilities as a parent.}

%------------------------------------------------

\Description{\MarginText{Teamwork}\textbullet \ \ Much of the work over the past few years has required collaboration with others, both within the research group and in other groups in other departments or universities.}

%------------------------------------------------

\vspace{1em} % Extra space between major sections

%----------------------------------------------------------------------------------------
%	WORK EXPERIENCE
%----------------------------------------------------------------------------------------

\noindent\spacedallcaps{Relevant Work Experience}\vspace{1em}

\NewEntry{2012-Present}{\MarginText{University of Leeds}\small\textbf{Post Graduate Demonstrator}}{\hfill School of Chemistry}

\Description{Consists of helping and marking the work of second year students in the undergraduate experimental labs and also helping with the teaching ofd the first year students in the ``Physics for Chemists'' module}

\NewEntry{2012,2011,2010}{\MarginText{University of Leeds}\small\textbf{Summer Research Student}}{\hfill Experimental Quantum Information}

\Description{Worked for two consecutive summers in the Experimental Quantum Information labs on development of a magnetic cardiogram, and in the summer of 2012 worked in the same labs helping a MSc student in the completion of his experimental work.}

\vspace{1em} % Extra space between major sections
%----------------------------------------------------------------------------------------
%	PUBLICATIONS
%----------------------------------------------------------------------------------------

\noindent\spacedallcaps{Publications}\vspace{1em}

\NewEntry{February 2015}{\MarginText{Physical Review A}\small\textbf{Coupled-Coherent States Approach for High Harmonic Generation}}

\Description{Authors: C. \textsc{Symonds}, ~J. \textsc{Wu}, ~M. \textsc{Ronto}, ~C. \textsc{Zagoya}, ~D. \textsc{Shalashilin}, ~C.F.~de M. \textsc{Faria}}

%------------------------------------------------

\NewEntry{In Preparation}{\MarginText{Journal of Physical Chemistry A}\small\textbf{Multidimentional quantum simulations of model systems and ab initio first principle quantum direct dynamics of ultrafast photochemistry with Multiconfigurational Ehrenfest approach.}}

\Description{Authors: D. \textsc{Makhov}, ~K \textsc{Saita}, ~C. \textsc{Symonds}, ~D. \textsc{Shalashilin}}

\NewEntry{In Preparation}{\MarginText{Journal of Chemical Physics}\small\textbf{Multiple Cloning corrections to the Multi-Configurational Ehrenfest approach - application to the Spin Boson Model}}

\Description{Authors: C. \textsc{Symonds}, ~D. \textsc{Shalashilin}}

%------------------------------------------------

\vspace{1em} % Extra space between major sections

%----------------------------------------------------------------------------------------
%	Funding and Awards
%----------------------------------------------------------------------------------------

\noindent\spacedallcaps{Funding and Awards}\vspace{1em}

\Description{2012\ \ $\cdotp$\ \ University of Leeds School of Chemistry Doctoral Training Grant}

\vspace{-0.5em} % Negative vertical space to counteract the vertical space between every \Description command

\Description{2011\ \ $\cdotp$\ \ Nuffield Summer Research Scholarship}

\vspace{-0.5em} % Negative vertical space to counteract the vertical space between every \Description command

\Description{2009\ \ $\cdotp$\ \ Derek Moody Award for Academic Excellence}

%------------------------------------------------

\vspace{1em} % Extra space between major sections

%----------------------------------------------------------------------------------------
%	References
%----------------------------------------------------------------------------------------
\newpage
\noindent\spacedallcaps{References}\vspace{1em}

% first column
\begin{multicols}{2}
	\noindent Prof. D. V. Shalashilin.\\
	School of Chemistry\\
	University of Leeds\\
	Woodhouse Lane\\
	Leeds\\
	LS2 9JT\\
	email:\href{mailto:D.Shalashilin@leeds.ac.uk}{D.Shalashilin@leeds.ac.uk}
	\vfill
	\columnbreak

	\noindent Prof. B. Varcoe\\
	School of Physics and Astronomy\\
	University of Leeds\\
	Woodhouse Lane\\
	Leeds\\
	LS2 9JT\\
	email:\href{mailto:B.Varcoe@leeds.ac.uk}{B.Varcoe@leeds.ac.uk}
\end{multicols}

\end{cv}

\end{document}
